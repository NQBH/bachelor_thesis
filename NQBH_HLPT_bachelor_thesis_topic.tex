\documentclass[oneside]{book}
\usepackage[backend=biber,natbib=true,style=alphabetic,maxbibnames=50]{biblatex}
\addbibresource{/home/nqbh/reference/bib.bib}
\usepackage[utf8]{vietnam}
\usepackage{tocloft}
\renewcommand{\cftsecleader}{\cftdotfill{\cftdotsep}}
\usepackage[colorlinks=true,linkcolor=blue,urlcolor=red,citecolor=magenta]{hyperref}
\usepackage{amsmath,amssymb,amsthm,enumitem,fancyvrb,float,graphicx,mathtools,minitoc,tikz}
\usetikzlibrary{angles,calc,intersections,matrix,patterns,quotes,shadings}
\usepackage{fancyhdr}
\pagestyle{fancy}
\fancyhf{}
\addtolength{\headheight}{0pt}% obsolete
\lhead{\scshape\small\chaptername~\thechapter}
\rhead{\itshape\small\nouppercase{\leftmark}}
\renewcommand{\chaptermark}[1]{\markboth{#1}{}}
\cfoot{\thepage}
\renewcommand{\headrulewidth}{0.5pt}
\renewcommand{\footrulewidth}{0pt}
\fancyheadoffset[RE,LO]{-0.0\textwidth}

\usepackage{textcase}

\makeatletter
\def\@makechapterhead#1{%
    \vspace*{50\p@}%
    {\parindent \z@ \centering\normalfont
        \ifnum \c@secnumdepth >\m@ne
        \if@mainmatter
        \huge\bfseries \MakeTextUppercase{\@chapapp}\space \thechapter
        \par\nobreak
        \vskip 20\p@
        \fi
        \fi
        \interlinepenalty\@M
        \huge \bfseries \MakeTextUppercase{#1}\par\nobreak
        \vskip 40\p@
}}
\def\@makeschapterhead#1{%
    \vspace*{50\p@}%
    {\parindent \z@ \centering
        \normalfont
        \interlinepenalty\@M
        \huge \bfseries  \MakeTextUppercase{#1}\par\nobreak
        \vskip 40\p@
}}
\makeatother


\DeclareMathSymbol{\mathinvertedexclamationmark}{\mathclose}{operators}{'074}
\DeclareMathSymbol{\mathexclamationmark}{\mathclose}{operators}{'041}

\makeatletter
\newcommand{\raisedmathinvertedexclamationmark}{%
    \mathclose{\mathpalette\raised@mathinvertedexclamationmark\relax}%
}
\newcommand{\raised@mathinvertedexclamationmark}[2]{%
    \raisebox{\depth}{$\m@th#1\mathinvertedexclamationmark$}%
}
\begingroup\lccode`~=`! \lowercase{\endgroup
    \def~}{\@ifnextchar`{\raisedmathinvertedexclamationmark\@gobble}{\mathexclamationmark}}
\mathcode`!="8000
\makeatother

\usepackage{sectsty}
\allsectionsfont{\sffamily}
\allowdisplaybreaks
\newtheorem{assumption}{Assumption}
\newtheorem{baitoan}{Bài toán}
\newtheorem{cauhoi}{Câu hỏi}
\newtheorem{conjecture}{Conjecture}
\newtheorem{corollary}{Corollary}
\newtheorem{dangtoan}{Dạng toán}
\newtheorem{definition}{Definition}
\newtheorem{dinhly}{Định lý}
\newtheorem{dinhnghia}{Định nghĩa}
\newtheorem{example}{Example}
\newtheorem{ghichu}{Ghi chú}
\newtheorem{goal}{Goal}
\newtheorem{hequa}{Hệ quả}
\newtheorem{hypothesis}{Hypothesis}
\newtheorem{intuition}{Intuition}
\newtheorem{lemma}{Lemma}
\newtheorem{luuy}{Lưu ý}
\newtheorem{nhanxet}{Nhận xét}
\newtheorem{notation}{Notation}
\newtheorem{note}{Note}
\newtheorem{principle}{Principle}
\newtheorem{problem}{Problem}
\newtheorem{proposition}{Proposition}
\newtheorem{question}{Question}
\newtheorem{remark}{Remark}
\newtheorem{theorem}{Theorem}
\newtheorem{vidu}{Ví dụ}
\usepackage[left=1cm,right=1cm,top=1.5cm,bottom=1.5cm]{geometry}
\def\labelitemii{$\circ$}
\DeclareRobustCommand{\divby}{%
    \mathrel{\vbox{\baselineskip.65ex\lineskiplimit0pt\hbox{.}\hbox{.}\hbox{.}}}%
}
\setlist[itemize]{leftmargin=*}
\setlist[enumerate]{leftmargin=*}
\newcommand{\genstirlingI}[3]{%
    \genfrac{[}{]}{0pt}{#1}{#2}{#3}%
}
\newcommand{\genstirlingII}[3]{%
    \genfrac{\{}{\}}{0pt}{#1}{#2}{#3}%
}
\newcommand{\stirlingI}[2]{\genstirlingI{}{#1}{#2}}
\newcommand{\dstirlingI}[2]{\genstirlingI{0}{#1}{#2}}
\newcommand{\tstirlingI}[2]{\genstirlingI{1}{#1}{#2}}
\newcommand{\stirlingII}[2]{\genstirlingII{}{#1}{#2}}
\newcommand{\dstirlingII}[2]{\genstirlingII{0}{#1}{#2}}
\newcommand{\tstirlingII}[2]{\genstirlingII{1}{#1}{#2}}

\title{Potential Research Topics for Bachelor Thesis Supervision\\Các Chủ Đề Nghiên Cứu Làm Khóa Luận Tốt Nghiệp Đại Học}
\author{Nguyễn Quản Bá Hồng\footnote{A scientist- {\it\&} creative artist wannabe, a mathematics {\it\&} computer science lecturer of Department of Artificial Intelligence {\it\&} Data Science (AIDS), School of Technology (SOT), UMT Trường Đại học Quản lý {\it\&} Công nghệ TP.HCM, Hồ Chí Minh City, Việt Nam.\\E-mail: {\sf nguyenquanbahong@gmail.com} {\it\&} {\sf hong.nguyenquanba@umt.edu.vn}. Website: \url{https://nqbh.github.io/}. GitHub: \url{https://github.com/NQBH}.}\and Huỳnh Lê Phú Trung\footnote{A mathematics {\it\&} computer science lecturer of Department of Artificial Intelligence {\it\&} Data Science (AIDS), School of Technology (SOT), UMT Trường Đại học Quản lý {\it\&} Công nghệ TP.HCM, Hồ Chí Minh City, Việt Nam.\\E-mail: {\sf trung.huynhlephu@umt.edu.vn}.}}
\date{\today}

\begin{document}
\maketitle
\setcounter{secnumdepth}{4}
\setcounter{tocdepth}{4}
\dominitoc % Initialization

%------------------------------------------------------------------------------%

\section*{Preface}
Các đề tài nghiên cứu tiềm năng cho Khóa Luận Tốt Nghiệp cho sinh viên Khoa Công Nghệ, UMT.

%------------------------------------------------------------------------------%

\newpage
\tableofcontents

%------------------------------------------------------------------------------%

\chapter{Preliminaries -- Kiến Thức Chuẩn Bị}
\minitoc
{\it Keywords.}
\begin{enumerate}
    \item Machine Learning (ML).
    \begin{enumerate}
        \item Supervised Learning.
        \item Unsupervised Learning.
        \item Reinforcement Learning (RL).
    \end{enumerate}
    \item Deep Learning (DL).
\end{enumerate}

%------------------------------------------------------------------------------%

\section{Mathematical Analysis {\it\&} Numerical Analysis -- Giải Tích Toán Học {\it\&} Giải Tích Số}
\textbf{\textsf{Resources -- Tài nguyên.}}
\begin{enumerate}
    \item NQBH. {\it Lecture Note: Mathematical Analysis \& Numerical Analysis -- Bài Giảng: Giải Tích Toán Học \& Giải Tích Số}.
    
    PDF: {\sc url}: \url{https://github.com/NQBH/advanced_STEM_beyond/blob/main/analysis/lecture/NQBH_mathematical_analysis_lecture.pdf}.
\end{enumerate}

%------------------------------------------------------------------------------%

\section{Combinatorics {\it\&} Graph Theory -- Tổ Hợp {\it\&} Lý Thuyết Đồ Thị}
\textbf{\textsf{Resources -- Tài nguyên.}}
\begin{enumerate}
    \item NQBH. {\it Lecture Note: Combinatorics \& Graph Theory -- Bài Giảng: Tổ Hợp \& Lý Thuyết Đồ Thị}.
    
    PDF: {\sc url}: \url{https://github.com/NQBH/advanced_STEM_beyond/blob/main/combinatorics/lecture/NQBH_combinatorics_graph_theory_lecture.pdf}.
\end{enumerate}

%------------------------------------------------------------------------------%

\section{Mathematical Optimization -- Toán Tối Ưu}
\textbf{\textsf{Resources -- Tài nguyên.}}
\begin{enumerate}
    \item NQBH. {\it Lecture Note: Mathematical Optimization -- Bài Giảng: Toán Tối Ưu}.
    
    PDF: {\sc url}: \url{https://github.com/NQBH/advanced_STEM_beyond/blob/main/optimization/lecture/NQBH_mathematical_optimization_lecture.pdf}.
\end{enumerate}

%------------------------------------------------------------------------------%

\section{Artificial Intelligence (AI) -- Trí Tuệ Nhân Tạo}
\textbf{\textsf{Resources -- Tài nguyên.}}
\begin{enumerate}
    \item \cite{Kutyniok2023}. {\it Gitta Kutyniok}.{\it The Mathematics of AI}.
    
    \item \cite{Kutyniok2024}. {\it Gitta Kutyniok}.{\it The Mathematics of Reliable AI}.
    
    \item \cite{Norvig_Russel2021}. {\sc Peter Norvig, Stuart Russell}. {\it Artificial Intelligence: A Modern Approach}. 4e.
\end{enumerate}

%------------------------------------------------------------------------------%

\section{Machine Learning -- Học Máy}
\textbf{\textsf{Resources -- Tài nguyên.}}
\begin{enumerate}
    \item \cite{Cho_ML_lecture}. {\sc Kyunghyun Cho}. {\it Machine Learning: a Lecture Note}. arXiv.
    
    \item \cite{Deisenroth_Faisal_Ong2024}. {\it Mathematics for Machine Learning}. 1e.
\end{enumerate}

%------------------------------------------------------------------------------%

\subsection{Artificial Neural Networks (ANNs) -- Mạng Nơron Nhân Tạo}
\textbf{\textsf{Resources -- Tài nguyên.}}
\begin{enumerate}
    \item \cite{Bach2024}. {\sc Francis Bach}. {\it Learning Theory from First Principles}. 1e.
    
    \item \cite{Mandic_Chambers2001}. {\sc Danilo P. Mandic, Jonathan A. Chambers}. {\it Recurrent Neural Networks for Prediction: Learning Algorithms, Architectures and Stability}. 1e.
\end{enumerate}

%------------------------------------------------------------------------------%

\section{Deep Learning -- Học Sâu}
\textbf{\textsf{Resources -- Tài nguyên.}}
\begin{enumerate}
    \item \cite{Bishop_Bishop2024}. {\sc Christopher M. Bishop, Hugh Bishop}. {\it Deep Learning: Foundations \& Concepts}.
    
    \item \cite{Chollet2021}. {\sc Fran\c{c}ois Chollet}. {\it Deep Learning with Python}. 2e.
    
    \item \cite{LeCun_Bengio_Hinton2015}. {\sc Yann LeCun, Yoshua Bengio, Geoffrey Hinton}. {\it Deep Learning}. Nature.
\end{enumerate}

%------------------------------------------------------------------------------%

\chapter{Combinatorial Neural Networks \& Optimization Problems in Graph Theory}
\minitoc

\begin{enumerate}
    \item {\it Keywords.} Combinatorial neural networks.
    \item {\it Student.} {\sc Phan Vĩnh Tiến [PVT].}
\end{enumerate}
\textbf{\textsf{Resources -- Tài nguyên.}}
\begin{enumerate}
    \item {\sc Alessandro Benfenati, Emilie Chouzenoux, Laurent Duval, Jean-Christophe Pesquet, Aurélie Pirayre}. {\it A review on graph optimization \& algorithmic frameworks}. [Research Report] LIGM - Laboratoire d'Informatique Gaspard-Monge.
    
    \item {\sc Quentin Cappart, Didier Ch\`etelat, Elias B. Khalil, Andrea Lodi, Christopher Morris, Petar Veli\v{c}kovi\'c}. {\it Combinatorial Optimization \& Reasoning with Graph Neural Networks}.
    
    \item {\sc Irwan Bello, Hieu Pham, Quoc V. Le, Mohammad Norouzi, Samy Bengio} (Google Brain). {\it Neural Combinatorial Optimization with Reinforcement Learning}. ICLR2017.
    
    \item {\sc Andoni I. Garmendia, Josu Ceberio, Alexander Mendiburu}. {\it Neural Combinatorial Optimization: a New Player in the Field}.
    
    \item \cite{Goldengorin2018}. {\sc Boris Goldengorin}. {\it Optimization Problems in Graph Theory}.
    
    \item \cite{Norvig_Russel2021}. {\sc Peter Norvig, Stuart Russell}. {\it Artificial Intelligence: A Modern Approach}. 4e.
\end{enumerate}

%------------------------------------------------------------------------------%

\chapter{Computer Music}
\minitoc
\begin{enumerate}
    \item {\it Keywords.} Automatic music transcription, music generation.
    \item {\it Student.} {\sc Võ Ngọc Trâm Anh [VNTA].}
\end{enumerate}
\textbf{\textsf{Resources -- Tài nguyên.}}
\begin{enumerate}
    \item \cite{Briot_Hadjeres_Pachet2020}. {\sc Jean-Pierre Briot, Ga\"etan Jadjeres, Fran\c{c}ois-David Pachet Pachet}. {\it Deep Learning Techniques for Music Generation}.
    
    \item \cite{Dubnov_Greer2023}. {\sc Shlomo Dubnov, Ross Greer}. {\it Deep \& Shallow: Machine Learning in Music \& Audio}.
    
    \item \cite{Horn_West_Roberts2022}. {\sc Michael S.  Horn, Melanie West, Cameron Roberts}. {\it Introduction to Digital Music with Python Programming: Learning Music with Code}. 1e.
    
    {\sf Comment.} Sách có hơi nhiều lỗi chính tả.
    \item \cite{Mueller2015,Mueller2021}. {\sc Meinard M\"{u}ller}. {\it Fundamentals of Music Processing -- Using Python \& Jupyter Notebooks}.
    
    {\sf Comment.} Mathematically rigorous enough $\Rightarrow$ Main reference.
\end{enumerate}
\textbf{\textsf{Research community -- Cộng đồng nghiên cứu.}}
\begin{enumerate}
    \item {\sc Meinard M\"uller}. \href{https://scholar.google.de/citations?user=uggxDWIAAAAJ&hl=en}{Google Scholar}.
\end{enumerate}

%------------------------------------------------------------------------------%

\section{Automatic Music Transcription (AMT)}
{\it Keywords.} 

%------------------------------------------------------------------------------%

\section{Music Generation}
{\it Keywords.} Stochastic, random Boltzmann machine (RBM).

%------------------------------------------------------------------------------%

\chapter{Computer Vision}
\minitoc
\textbf{\textsf{Resources -- Tài nguyên.}}
\begin{enumerate}
    \item Associate Prof. {\sc Lý Quốc Ngọc}. {\it Lecture: Introduction to Image Processing \& Applications -- Bài Giảng: Nhập Môn Xử Lý Ảnh \& Ứng Dụng}.
    \item {\sc David Tschumperle, Christophe Tilman, Vincent Barra}. {\it Digital Image Processing with C++: Implementing Reference Algorithms with the CImg Library}.
    \item {\sc Mark S. Nixon, Alberto S. Aguado}. {\it Feature Extraction \& Image Processing for Computer Vision}. 4e.
    \item {\sc Manas Kamal Bhuyan}. {\it Computer Vision \& Image Processing Fundamentals \& Applications}.
    \item {\sc Rafael C. Gonzalez, Richard E. Woods}. {\it Digital Image Processing}. 4e.
    \item {\sc Martin McBridge}. {\it Image Processing in Python}.
\end{enumerate}

%------------------------------------------------------------------------------%

\section{Handwritten Digit Classification}

%------------------------------------------------------------------------------%

\chapter{Scheduling Problems}
\minitoc
\begin{enumerate}
    \item {\it Keywords.} Deterministic scheduling problem, stochastic scheduling problems.
    \item {\it Student.} {\sc Nguyễn Ngọc Thạch [NNT].}
\end{enumerate}
\textbf{\textsf{Resources -- Tài nguyên.}}
\begin{enumerate}
    \item \cite{Pinedo2022}. {\sc Michael L. Pinedo}. {\it Scheduling: Theory, Algorithms, \& Systems}.
\end{enumerate}

%------------------------------------------------------------------------------%

\chapter{Miscellaneous}
\minitoc

%------------------------------------------------------------------------------%

\section{Linux}
\textbf{\textsf{Resources -- Tài nguyên.}}
\begin{enumerate}
    \item \cite{Shotts2019}. {\sc William Shotts}. {\it The Linux Command Line: A Complete Introduction}. 2nd.
\end{enumerate}

%------------------------------------------------------------------------------%

\section{Contributors}

\begin{enumerate}
    \item {\sc Võ Ngọc Trâm Anh [VNTA].}
    \item {\sc Sơn Tân [ST].}
    \item {\sc Nguyễn Ngọc Thạch [NNT].}
    \item {\sc Phan Vĩnh Tiến [PVT].}
\end{enumerate}

%------------------------------------------------------------------------------%

\printbibliography[heading=bibintoc]

\end{document}